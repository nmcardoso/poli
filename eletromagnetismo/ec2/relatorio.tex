\documentclass{aleph-revista}

\usepackage{tikz}           
\usepackage{aleph-comandos} 
\usepackage{multicol}    
\usepackage{indentfirst}
\usepackage{cleveref}
\usepackage{circuitikz}
\usepackage{graphicx}
\usepackage{listings}
\usepackage{xcolor}
\usepackage{tikz-dimline}
\usepackage{steinmetz}
\usetikzlibrary{arrows,calc,patterns,decorations.markings,arrows.meta,quotes}

\definecolor{commentgreen}{RGB}{2,112,10}
\definecolor{eminence}{RGB}{108,48,130}
\definecolor{weborange}{rgb}{0.58,0,0.82}
\definecolor{frenchplum}{RGB}{129,20,83}

\definecolor{tabred}{RGB}{214, 39, 40}
\definecolor{tabblue}{RGB}{31, 119, 180}
\definecolor{tabgreen}{RGB}{44, 160, 44}

\lstset {
    language=Python,
    frame=tb,
    tabsize=4,
    showstringspaces=false,
    numbers=left,
    upquote=true,
    commentstyle=\color{commentgreen},
    keywordstyle=\color{eminence},
    stringstyle=\color{red},
    basicstyle=\small\ttfamily,
    emphstyle={\color{blue}},
    escapechar=\&,
    classoffset=1,
    otherkeywords={>,<,.,;,-,!,=,~},
    morekeywords={>,<,.,;,-,!,=,~},
    keywordstyle=\color{weborange},
    classoffset=0,
}
% \graphicspath{{figures/}}

\titulo{Exercício Computacional 02}
\tituloingles{Método dos Momentos}

\autor{%
  Natanael Magalhães Cardoso\textsuperscript{1}
}

\institucion{
\textsuperscript{1}%
  n$^{o}$ USP: 8914122
}

\fecha{31 de julho de 2021}

% \abstract{
%   Este relatório mostra uma solução computacional de um problema do eletromagnetismo com um algorítmo de diferenciação numérica, o Método das Diferenças Finitas. Aqui são expostos dois problemas onde as equipotenciais, as linhas de campo elétrico e o valor da resistência são calculados.
% }

\begin{document}
\membrete

\vspace{1em}


\section{Matriz de Capacitâncias $[C']_3$}
% \subsection{Hipóteses}
% \begin{enumerate}
%   \item Pelos valores da Tabela \ref{tab:dados}, a seguinte relação é estabelecida: $2a \ll \min(d, D, \mathcal{H}_k)$, onde $a$ é o raio de cada linha de transmissão, $\mathcal{H}_k$ é a altura da linha $k$ em relação ao solo, $D$ é a distância horizontal entre as linhas e $d$ é a distância vertical entre as linhas. Como o diâmetro da linha é $\approx 50$ vezes menor que a menor grandeza mensionada, as linhas de transmissão serão simplificados por linhas de carga imagens (retas) usando o método das imagens.
%   \item Um fio descarregado tem influência desprezível sobre o potencial produzido pelo outro.
%   \item As linhas de carga são uniformimente carregadas com cargas $Q_i$ e $-Q_i$.
% \end{enumerate}

\subsection{Inspeção da representação computacional do sistema}
A implementação deste algorítmo representa a geometria do problema por uma matriz. Antes de executá-lo para calcular algum valor, é importante verificar se a geometria do sistema está devidamente representada pela matriz. O diagrama da Figura \ref{fig:matrix} mostra a forma geral da matriz, no painel à esquerda, e os limites das posições dos corpos, nos painéis à direita.


\newpage

\begin{figure}[!h]
  \centering
  \includegraphics[width=\textwidth]{figures/matrix_plot.pdf}
  \caption{Visualização da matriz do sistema. O painél da esquerda mostra a geometria da matriz do sistema e os da direita mostram os corpos 1, 2 e 3 (de baixo para cima) em escala aumentada. Os pontos azuis indicam os cilindros de raio $b=1\cdot 10^{-4}$.}
  \label{fig:matrix}
\end{figure}

\newpage


\subsection{Calibração do raio dos cilindros de discretização}
Nesta implementação do algorítmo do Métodos dos Momentos, os corpos são discretizados por pequenos cilindros. A acurácia do resultado numérico calculado (da aproximação calculada) pelo algorítmo é tão melhor quanto menor for o tamanho deste raio. Sendo assim, uma calibração do raio foi feita para validar que a aproximação calculada seja suficientemente próxima ao valor real. Nesta calibração, é ajustado um valor para o raio tal que o erro percentual, calculado a partir da equação \eqref{eq:erro}, entre o valor estimado e o teórico seja menor que 0.5\%. A Figura \ref{fig:calibration} mostra este erro percentual em função do raio dos cilindros de discretização. O menor erro percentual obtido foi de 0.0205\% para o raio de discretização $b=5\cdot 10^{-5}$.

\begin{equation}\label{eq:erro}
  \text{Erro percentual} = \left|\frac{C_{teo} - C_{calc}}{C_{teo}}\right|
\end{equation}

\begin{figure}[!h]
  \centering
  \includegraphics[width=0.95\textwidth]{figures/calibration}
  \caption{Gráfico log-log (com o eixo $x$ invertido) do erro percentual entre o valor da capacitância no corpo 1 calculado pelo algorítmo e o valor teórico em função do raio do cilindro de discretização.}
  \label{fig:calibration}
\end{figure}




\subsection{Resultado do algorítmo}

A matriz de capacitâncias, calculada pelo algorítmo com um raio de discretização ${b=5\cdot 10^{-5}}$, é mostrada na equação \eqref{eq:cap-a}.

\begin{equation}\label{eq:cap-a}
  [C']_3 =
  \begin{bmatrix}
    23.394  & -20.812 & -0.869 \\
    -20.812 & 29.882  & -3.703 \\
    -0.869  & -3.703  & 9.699
  \end{bmatrix}
  \text{nF}
\end{equation}

\section{Diagrama Elétrico de $[C']_3$}

O modelo de circuito elétrico com parâmetros concentrados é obtido de $[C']_3$, da equação \eqref{eq:cap-a}, onde a capacitância $C_{ki}$ é calculada pela relação \eqref{eq:Cki}.

\begin{equation}\label{eq:Cki}
  C_{ki} =
  \begin{cases}
    \displaystyle \sum_{j=1}^N C_{kj}', & \text{se } i=0      \\
    \displaystyle -C_{ki}',             & \text{se } i \neq k
  \end{cases}
\end{equation}

Das equações \eqref{eq:cap-a} e \eqref{eq:Cki}, calcula-se
\begin{align*}
  C_{10} & = 1.714 \textrm{ nF} & C_{12} = C_{21} & = 20.812 \textrm{ nF} \\
  C_{20} & = 5.367 \textrm{ nF} & C_{13} = C_{31} & = 0.869 \textrm{ nF}  \\
  C_{30} & = 5.128 \textrm{ nF} & C_{23} = C_{32} & = 3.703 \textrm{ nF}
\end{align*}

A partir dos valores acima, pode-se montar o diagrama do circuito elétrico correspondente ao sistema, como mostrado na Figura \ref{fig:circ-b}.

\begin{figure}[!h]
  \centering
  \begin{circuitikz}[scale=1.6]
    \draw (0,0) node[below left](N1){1}
    to[C=20.812 nF, *-] (3,0) node[below left](N2){2}
    to[C=3.703 nF, *-*] (6,0) node[below left](N3){3}

    (N1.north east) -- ++(0,1.5) to[C=0.869 nF] ++(6,0) -- (N3.north east)

    (N1.north east) to[C=1.714 nF] ++(0,-2.5) node[ground]{}
    (N2.north east) to[C=5.367 nF] ++(0,-2.5) node[ground]{}
    (N3.north east) to[C=5.128 nF] ++(0,-2.5) node[ground]{}
    ;
  \end{circuitikz}
  \caption{Diagrama elétrico do sistema.}
  \label{fig:circ-b}
\end{figure}


\section{Matriz de Capacitâncias $[C']_2$}
\label{section:c}

A matriz de capacitâncias $[C']_2$, calculada removendo o corpo 2 do sistema definido anteriormente, está representada na equação \eqref{eq:cap-c}.

\begin{equation}\label{eq:cap-c}
  [C']_2 =
  \begin{bmatrix}
    8.747  & -3.389 \\
    -3.389 & 9.218  \\
  \end{bmatrix}
  \text{nF}
\end{equation}

Usando a equação \eqref{eq:Cki} para o cálculo da capacitância com parâmetros concentrados e a equação \eqref{eq:cap-c}, calcula-se
\begin{align*}
  C_{10}          & = 5.358 \textrm{ nF} \\
  C_{30}          & = 5.829 \textrm{ nF} \\
  C_{13} = C_{31} & = 3.389 \textrm{ nF} \\
\end{align*}

Assim, o diagrama elétrico do circuito é mostrado na Figura \ref{fig:circ-c}.

\begin{figure}[!h]
  \centering
  \begin{circuitikz}[scale=1.6]
    \draw (0,0) node[below left](N1){1}
    to[C=3.389 nF, *-*] (3,0) node[below left](N2){3}

    (N1.north east) to[C=5.358 nF] ++(0,-2.5) node[ground]{}
    (N2.north east) to[C=5.829 nF] ++(0,-2.5) node[ground]{}
    ;
  \end{circuitikz}
  \caption{Diagrama elétrico do sistema.}
  \label{fig:circ-c}
\end{figure}

Os valores das capacitâncias entre os corpos 1 e 3 e entre os corpos 1 e 3 e a terra são alterados com a remoção do corpo 2. Ambas as capacitâncias aumentam, mas a capacitância do corpo 1, que estava mais perto do corpo 2, aumenta muito mais.

\newpage
\section{Gerador Thévenin}

\subsection{Caso 1: Corpo 2 aterrado}
O gerador de Thévenin para este caso pode ser calculado a partir do diagrama elétrico da Figura \ref{fig:circ-b}. Na Figura \ref{fig:circ-b1}, os ramos em vermelho serão desconsiderador por conta do curto circuito, em azul.

\begin{figure}[!h]
  \centering
  \begin{circuitikz}[scale=1.3]
    \draw (0,0) node[above left](N1){1};
    \draw[color=tabred] (0,0) to[C=20.812 nF, o-] (3,0) node[below left, color=black](N2){2};
    \draw (3,0) to[C=3.703 nF, *-*] (6,0) node[below left](N3){3};

    \draw[color=tabred] (0,0) -- ++(0,1.5) to[C=0.869 nF] ++(6,0) -- (N3.north east);

    \draw[color=tabblue] (N2.north east) to[short] ++(0,-3);
    \draw (N3.north east) to[C=5.128 nF] ++(0,-1.5) to[sV, v=$V_s$, american, voltage dir=noold] ++(0,-1.5) node[ground](G){};
    \draw (G) to[short, -o] ++(-6,0) node[](A){};
    % \draw (0,0) to[C=1.714 nF] (A) node[]{};
    \path (A) edge[->, >=Stealth, bend left] node[left]{$V_{th}$} (0,0);
  \end{circuitikz}
  \caption{Diagrama elétrico do sistema.}
  \label{fig:circ-b1}
\end{figure}

O diagrama elétrico após as simplificações é mostrado na Figura \ref{fig:circ-b2}.

\begin{figure}[!h]
  \centering
  \begin{circuitikz}[scale=1.3]
    \draw (3,0) node[above left, color=black](N2){2};
    \draw (3,0) to[C=3.703 nF, o-*] (6,0) node[below left](N3){3};

    \draw (N3.north east) to[C=5.128 nF] ++(0,-1.5) to[sV, v=$V_s$, american, voltage dir=noold] ++(0,-1.5) node[ground](G){};
    \draw (G) to[short, -o] ++(-3,0) node[](A){};
    \path (A) edge[->, >=Stealth, bend left] node[left]{$V_{th}$} (3,0);
  \end{circuitikz}

  \caption{Diagrama elétrico simplificado.}
  \label{fig:circ-b2}
\end{figure}


% \begin{figure}[!h]
%   \centering
%   \begin{circuitikz}[scale=1.3]
%     \draw (3,0) node[above left, color=black](N2){2};
%     \draw (3,0) to[C=3.703 nF, o-*] (6,0) node[below left](N3){3};

%     \draw (N3.north east) to[C=5.128 nF] ++(0,-2) node[ground](G){};
%     \draw (G) to[short, -o] ++(-3,0) node[](A){};
%     \path (A) edge[->, >=Stealth, bend left] node[left]{$V_{th}$} (3,0);
%   \end{circuitikz}

%   \caption{Diagrama elétrico do sistema.}
%   \label{fig:circ-b3}
% \end{figure}

Calculando a impedância de cada bipolo e a impedância equivalente
\begin{equation}
  Z_{23} = \frac{1}{j\omega C_{23}} = -j716.33 = 716.33\phase{-90^{\circ}}\text{ k}\Omega
\end{equation}
\begin{equation}
  Z_{30} = \frac{1}{j\omega C_{30}} = -j 517.27 = 517.27\phase{-90^{\circ}}\text{ k}\Omega
\end{equation}
\begin{equation}
  Z_{eq} = Z_{23} + Z_{30} = -j1233.6 \text{ k}\Omega = 1233.6\phase{-90^{\circ}}\text{ k}\Omega
\end{equation}

Calculando a corrente do circuito e a tensão de Thévenin, considerando que o potencial $a$ é a tensão de saída do bipolo $C_{23}$
\begin{equation}
  I = \frac{V_s}{Z} = \frac{14\text{k}\phase{0^{\circ}}}{Z_{eq}} = \frac{14\phase{0^{\circ}}}{1233.6\phase{-90^{\circ}}} = 11.31\phase{90^{\circ}}\ \text{mA}
\end{equation}
\begin{equation}
  V_{ab} = V_{th} = I \cdot Z = (11.31\phase{90^{\circ}}\ \text{mA})(716.33\phase{-90^{\circ}}\text{ k}\Omega) = 8101.69\phase{0^{\circ}} \text{ V}
\end{equation}

Calculando a corrente de curto circuito e a impedância de Thévenin, considerando que o tensão medida entre os terminais $a$ e $b$ de Thévenin já está em curto circuito e não possui queda de tensão
\begin{equation}
  I_{sc} = I = 11.31\phase{90^{\circ}}\ \text{mA}
\end{equation}
\begin{equation}
  Z_{th} = \frac{V_{th}}{I_{sc}} = 716.33\phase{-90^{\circ}} = -j716.33\ \text{k}\Omega
\end{equation}

O Diagrama do Equivalente de Thévenin para o circuito é mostrado na Figura \ref{fig:th-1}. Como a impedância de Thévenin $Z_{th} = -j716.33\ \text{k}\Omega$ possui comportamento capacitivo, ela foi representada por um capacitor.

\begin{figure}[!h]
  \centering
  \begin{circuitikz}[scale=1.3]
    \draw (3,0) node[above left, color=black](N2){};
    \draw (3,0) node[above]{$b$} to[C=$-j716.33\ \Omega$, o-] (6,0) node[below left](N3){};

    \draw (N3.north east) to[sV, v=$8.1\phase{0^{\circ}}\text{ kV}$, american, voltage dir=noold] ++(0,-2) node[ground](G){};
    \draw (G) to[short, -o] ++(-3,0) node[label=$a$](A){};
  \end{circuitikz}
  \caption{Equivalente Thévenin}
  \label{fig:th-1}
\end{figure}


\subsection{Casos 2 e 3: Corpo 2 isolado/ausente}

O circuito usado para calcular o equivalente Thévenin para o corpo 2 ausente será baseado no diagrama gerado na Seção \ref{section:c} e está ilustrado na Figura \ref{fig:circ-c1}.

\begin{figure}[!h]
  \centering
  \begin{circuitikz}[scale=1.6]
    \draw (0,0) node[above left](N1){1}
    to[C=3.389 nF, o-*] (3,0) node[below left](N2){3}

    % (N1.north east) to[C=5.358 nF] ++(0,-2.5) node[ground]{}
    (N2.north east) to[C=5.829 nF] ++(0,-2.5) node[ground](G){}
    ;
    \draw (G) to[short, -o] ++(-3,0) node[](A){};
    \path (A) edge[->, >=Stealth, bend left] node[left]{$V_{th}$} (0,0);
  \end{circuitikz}
  \caption{Diagrama elétrico do sistema.}
  \label{fig:circ-c1}
\end{figure}

Calculando a impedância de cada bipolo e a impedância equivalente
\begin{equation}
  Z_{13} = \frac{1}{j\omega C_{13}} = -j782.7 = 782.7\phase{-90^{\circ}}\text{ k}\Omega
\end{equation}
\begin{equation}
  Z_{30} = \frac{1}{j\omega C_{30}} = -j 455.07 = 455.07\phase{-90^{\circ}}\text{ k}\Omega
\end{equation}
\begin{equation}
  Z_{eq} = Z_{13} + Z_{30} = -j1237.77 \text{ k}\Omega = 1237.77\phase{-90^{\circ}}\text{ k}\Omega
\end{equation}

% Calculando a corrente do circuito e a tensão de Thévenin, considerando que o potencial $a$ é a tensão de saída do bipolo $C_{13}$
% \begin{equation}
%   I = \frac{V_s}{Z} = \frac{14\text{k}\phase{0^{\circ}}}{Z_{eq}} = \frac{14\phase{0^{\circ}}}{1233.6\phase{-90^{\circ}}} = 11.35\phase{90^{\circ}}\ \mu\text{A}
% \end{equation}
% \begin{equation}
%   V_{ab} = V_{th} = I \cdot Z = (11.35\phase{90^{\circ}}\ \mu\text{A})(782.7\phase{-90^{\circ}}\text{ k}\Omega) = 8.609\phase{0^{\circ}} \text{ V}
% \end{equation}

% Calculando a corrente de curto circuito e a impedância de Thévenin, considerando que o tensão medida entre os terminais $a$ e $b$ de Thévenin já está em curto circuito e não possui queda de tensão
% \begin{equation}
%   I_{sc} = I = 11.35\phase{90^{\circ}}\ \mu\text{A}
% \end{equation}
% \begin{equation}
%   Z_{th} = \frac{V_{th}}{I_{sc}} = 716.33\phase{-90^{\circ}} = -j716.33\ \text{k}\Omega
% \end{equation}

% O Diagrama do Equivalente de Thévenin para o circuito é mostrado na Figura \ref{fig:th-1}. Como a impedância de Thévenin $Z_{th} = -j716.33\ \text{k}\Omega$ possui comportamento capacitivo, ela foi representada por um capacitor.

% \begin{figure}[!h]
%   \centering
%   \begin{circuitikz}[scale=1.3]
%     \draw (3,0) node[above left, color=black](N2){};
%     \draw (3,0) node[above]{$b$} to[C=$-j716.33\ \text{k}\Omega$, o-] (6,0) node[below left](N3){};

%     \draw (N3.north east) to[sV, v=$7.88\phase{0^{\circ}}\text{ V}$, american, voltage dir=noold] ++(0,-2) node[ground](G){};
%     \draw (G) to[short, -o] ++(-3,0) node[label=$a$](A){};
%   \end{circuitikz}
%   \caption{Equivalente Thévenin}
%   \label{fig:th-1}
% \end{figure}





\end{document}
