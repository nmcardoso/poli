\documentclass[a4, 12pt]{lab-report}

\usepackage{mathpazo} % MathPazo
% \usepackage{times} % Times New Roman
% \usepackage{helvet} % Helvetica
% \renewcommand{\familydefault}{\sfdefault}
\usepackage{lipsum}
\usepackage{karnaugh-map}

\newcommand{\n}[1]{\overline{#1}}


% Informações da capa
% ~~> Mudar a cada experimento <~~
\setExpNumber{2}
\setExpTitle{Mapas de Carnô}
\setDate{10/03/2022}

% Não precisa mudar
\setTeacher{Glauber de Bona}
\setBancada{B3}
\setTurma{10}
\setMemberOne{Natanael Magalhães Cardoso}{8914122}
\setMemberTwo{Renato Naves Fleury}{11805269}


% Início do texto
\begin{document}
\frontpage

\section{Introdução}
$$
  \overline{\overline{X} + \overline{Y}}
$$
\lipsum{1}

\section{Objetivo}
\lipsum{1}

\section{Planejamento}
\lipsum{1}



\begin{figure}
\centering
\begin{karnaugh-map}[4][4][1][$B_1B_0$][$A_1A_0$]
\minterms{0,2,3,6,7,10,12,13,15}
\autoterms[0]
\implicant{3}{6}
\implicant{12}{13}
\implicant{7}{15}
\implicantedge{0}{0}{2}{2}
\implicantedge{2}{2}{10}{10}
\end{karnaugh-map}
\caption{Mapa de Karnaugh}
\end{figure}

\begin{equation}
  Z = \n{A_1} \cdot \n{A_0} \cdot \n{B_0} \;+\;
  \n{A_1} \cdot B_1 \;+\;
  A_0 \cdot B_1 \cdot B_0 \;+\;
  A_1 \cdot A_0 \cdot B_1 \;+\;
  \n{A_0} \cdot B_1 \cdot \n{B_0}
\end{equation}


\end{document}
