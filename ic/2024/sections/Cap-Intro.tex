\section{Introdução}
\label{cap:introducao}

Com o avanço das tecnologias de captura e armazenamento de imagens em larga escala, levantamentos astronômicos modernos, como o Sloan Digital Sky Survey (SDSS; Seção \ref{sec:sdss}) e o DESI Legacy Survey (Seção \ref{sec:legacy}), têm produzido catálogos de imagens da ordem de dezenas de bilhões de objetos celestes. Logo, torna-se crescente a demanda por ferramentas eficientes e precisas para a análise de grandes volumes de dados astronômicos de forma automatizada.

Além disso, para a próxima década, está previsto o início da operação de novos telescópios projetados para ter uma capacidade de geração de dados jamais vista na astronomia, como o Giant Magellan Telescope (GMT; \citealp{gmt}) e o Vera Rubin Observatory -- Legacy Survey of Space and Time (LSST; \citealp{lsst}). Esse aumento exponencial na quantidade de dados causarão novos desafios à pesquisa em astronomia, uma vez que a análise manual tornar-se-á inviável e métodos convencionais de classificação e busca não conseguem lidar com a complexidade e a diversidade das características visuais dos objetos celestes.

Por isso, o desenvolvimento de um sistema de busca baseado em similaridade visual é fundamental para identificar e classificar objetos com propriedades morfológicas e estruturais semelhantes, como galáxias espirais, elípticas e irregulares, que possuem características específicas associadas a sua formação e evolução. Esses aspectos são fundamentais para entender a estrutura do universo e os processos físicos que governam a formação das galáxias. O sistema de busca por similaridade visual, desenvolvido neste trabalho, permite que os astrônomos realizem comparações entre objetos visualmente similares de forma automática, facilitando o estudo de fenômenos astronômicos complexos e a identificação de padrões em grandes amostras de dados. Essa capacidade de análise comparativa é particularmente valiosa para estudos de lentes gravitacionais, interações galácticas e evolução cósmica.

O objetivo deste projeto é desenvolver um sistema inteligente de busca por similaridade visual que permita a pesquisadores realizar consultas visuais rápidas e precisas em grandes bancos de dados astronômicos, utilizando uma forma automática de inspeção visual. Esse sistema deve ser capaz de identificar objetos morfologicamente semelhantes com base em suas propriedades visuais, promovendo uma análise detalhada e comparativa dos dados astronômicos

Inicialmente, na etapa de concepção, realizou-se uma investigação das principais características necessárias para que o sistema tivesse o impacto científico almejado com a especição dos requisitos funcionais (Seção \ref{sec:req-funcionais}) e não funcionais (Seção \ref{sec:req-nao-funcionais}).

Em seguida, na etapa de desenvolvimento, realizou-se uma análise dos levantamentos astronômicos disponíveis para selecionar os dados mais relevantes para o treinamento e avaliação do sistema (Seção \ref{sec:aquisicao}). Esses levantamentos foram escolhidos devido à sua abrangência, qualidade e disponibilidade pública. Em seguida, as imagens foram baixadas (Seção \ref{sec:aquisicao-stamps}) com o devido ajuste do campo de visão angular (Seção \ref{sec:aquisicao-fov}). O conjuto de dados foi segmentado por características físicas, assegurando que as amostras de treinamento, validação e teste fossem representativas e com menos viés possível (Seção \ref{sec:aquisicao-descricao}).

Posteriormente, foi implementado o modelo de aprendizado profundo, desenvolvido com arquiteturas modernas (Seção \ref{sec:arch}). O treinamento do modelo foi otimizado por meio de técnicas como aumento de dados (Seção \ref{sec:modelo-dataaug}) e ajuste automático de hiperparâmetros (Seção \ref{sec:modelo-treinamento}). Além disso, uma função de custo customizada (Seção \ref{sec:modelo-loss}) foi projetada para atendender aos requisitos específicos dos dados de treinamento utilizado.

Desta forma, foi feita a integração do modelo desenvolvido com um sistema de informação escalável. Para isso, foi desenvolvido um backend, responsável pelo gerenciamento das consultas de similaridade visual (Seção \ref{sec:si-backend}) e pela indexação eficiente dos embeddings gerados pelo modelo e de dados georreferenciados (Seção \ref{sec:si-tecnologias}). O frontend, desenvolvido com uma framework de programação reativa e com gerenciamento de estados declarativo, foi projetado com foco na usabilidade, oferecendo uma interface intuitiva (Seção \ref{sec:si-frontend}). A arquitetura do sistema foi complementada pelo uso de bancos de dados relacionais otimizados e pipelines de DevOps (Seção \ref{sec:si-devops}) para implantação e manutenção contínuas.

Por fim, o sistema foi validado por meio de experimentos quantitativos e qualitativos, utilizando métricas de avaliação de modelos (Seção \ref{sec:res-teste}) e de sistemas de recuperação de imagens (Seção \ref{sec:res-ret}). Além disso, testes em cenários reais foram conduzidos para demonstrar a aplicabilidade prática do sistema, incluindo buscas por objetos celestes específicos em grandes volumes de dados (Seção \ref{sec:res-ret}).

