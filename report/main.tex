\documentclass[12pt]{article}
\usepackage{extsizes}
\usepackage[brazil]{babel}
\usepackage{tikz}
\usepackage{graphicx}
\usepackage{caption}
\usepackage{subcaption}
\usepackage{stanli}
\usepackage{siunitx}
\usepackage{mathtools}
\usepackage{quotes}
\usepackage{amsmath}
\usepackage{empheq}
\usepackage{fancyhdr}
\usepackage{amsthm}
\usepackage{booktabs}

\begin{document}

\textbf{Dúvida sobre o projeto}

\hspace{10cm}

\begin{minipage}{.45\textwidth}
  \includegraphics[width=\textwidth]{fig/rb-suspension.png}
\end{minipage}%
\begin{minipage}{.45\textwidth}
  \includegraphics[width=\textwidth]{fig/rb.jpg}
\end{minipage}

\hspace{4cm}

O meu trabalho é sobre o sistema de suspensão rocker-bogie. Eu fiz este diagrama abaixo, mas tem algo errado com as equações.

\hspace{4cm}

\begin{figure}[h!]
  \input{fig/diagrama1.tex}
\end{figure}

\begin{align*}
  \sum F_H = 0 \;&\Rightarrow\; H_A + H_B + H_C = 0\\
  \sum F_V = 0 \;&\Rightarrow\; V_A + V_B + V_C - P = 0\\
  \sum M^{(A)} = 0 \;&\Rightarrow\; \frac{L}{2}V_A + LV_C - \frac{LP}{2} = 0\\
  \sum M^{(B)} = 0 \;&\Rightarrow\; \frac{-L}{2}V_A + \frac{L}{2}V_C = 0\\
  \sum M^{(C)} = 0 \;&\Rightarrow\; -LV_A - \frac{-L}{2}V_B + \frac{LP}{2} = 0\\
\end{align*}

\begin{figure}[h!]
  \centering
  \begin{minipage}{.5\textwidth}
    \begin{tikzpicture}
  \begin{scope}
    \point{a}{0}{0};
    \point{e}{2.5}{2.5};

    \beam{2}{a}{e};

    \support{1}{a};

    \hinge{1}{e};

    \load{1}{e}[90];

    \dnotation{1}{e}{P}[yshift=9mm,right=1mm];
    \dnotation{1}{a}{A}[above left];
    \dnotation{1}{e}{E}[below=2mm];

    \load{3}{e}[315][90][.8];
    \draw[->,thick] (e) -- (3.5,3.5);
    \draw[->,thick] (e) -- (3.5,1.5);

    \dnotation{1}{e}{$M_E$}[right=9mm];
    
  \end{scope}

  
\end{tikzpicture}
    \hfill
  \end{minipage}%
  \begin{minipage}{.5\textwidth}
    \hfill
    \begin{tikzpicture}
  \begin{scope}
    \point{c}{2.5}{0};
    \point{d}{0}{2.5};

    \beam{2}{c}{d};

    \support{1}{c};

    \hinge{1}{d};

    \dnotation{1}{c}{C}[above right];
    \dnotation{1}{d}{D}[above right];

    \draw[->,thick] (d) -- (-1,3.5);
    \draw[->,thick] (d) -- (-1,1.5);
    \load{2}{d}[135][90][.8]

    \dnotation{1}{d}{$M_D$}[left=9mm];
  \end{scope}
\end{tikzpicture}
  \end{minipage}
\end{figure}

Considerando os pontos D e E como articulações, eu consigo essas novas equações (que eu acho que estão erradas).

\begin{align*}
  \sum M^{(D)} = 0 \;&\Rightarrow\; \frac{L}{4}V_C + \frac{y}{2}H_C + \underbrace{M_D}_{0} = 0\\
  \sum M^{(E)} = 0 \;&\Rightarrow\; \frac{-L}{2}V_A + yH_A + \underbrace{M_E}_{0} = 0
\end{align*}

O sistema formado por essas seis equações não gera o número de equações LI suficientes para 
determinar os valores das 6 incógnitas, por isso acredito que as equações que escrevi estão erradas.
Gostaria de saber o que estou fazendo errado ou se existe uma forma melhor de fazer isso. 
Não sei se é melhor aplicar o método do equilíbrio dos nós.

\begin{empheq}[left=\empheqlbrace]{align}
  &\sum F_H = 0 \;\;\Rightarrow\;\; H_A + H_B + H_C = 0\\
  &\sum F_V = 0 \;\;\Rightarrow\;\; V_A + V_B + V_C - P = 0\\
  &\sum M^{(A)} = 0 \;\;\Rightarrow\;\; \frac{L}{2}V_A + LV_C - \frac{L}{2}P = 0\\
  &\sum M^{(B)} = 0 \;\;\Rightarrow\;\; -\frac{L}{2}V_A + \frac{L}{2}V_C = 0\\
  &\sum M^{(C)} = 0 \;\;\Rightarrow\;\; -LV_A - \frac{L}{2}V_B + \frac{L}{2}P = 0
\end{empheq}

\end{document}