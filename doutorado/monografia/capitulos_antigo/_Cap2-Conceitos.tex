\chapter{Aspectos Conceituais}
\label{cap:conceitos}

\section{Morfologia}

A morfologia, no contexto astronômico, refere-se ao estudo da forma, estrutura e aparência de objetos celestes, especialmente galáxias. A análise morfológica permite a classificação desses objetos com base em suas características visuais, como o formato, o brilho superficial, a presença de braços espirais ou a distribuição de estrelas. O estudo da morfologia é crucial para a compreensão da formação, evolução e dinâmica das galáxias e outros corpos celestes, sendo uma ferramenta central na astrofísica observacional e na cosmologia.
Classificação Morfológica de Galáxias

A classificação morfológica de galáxias remonta ao trabalho pioneiro de Edwin Hubble na década de 1920, que propôs um esquema de classificação em três categorias principais: galáxias elípticas, espirais e irregulares. Esse esquema, conhecido como Diagrama de Hubble ou Diapasão de Hubble, ainda é amplamente utilizado, embora tenha sido refinado e expandido com o advento de novas observações e levantamentos astronômicos.

Galáxias Elípticas: São galáxias com formas suaves e elipsoides, sem características estruturais notáveis, como braços espirais ou discos. A morfologia elíptica indica uma distribuição estelar aproximadamente esferoidal e pouco gás ou poeira, sugerindo que essas galáxias passaram por fusões galácticas ou processos de formação estelar no passado distante. Sua classificação varia de E0 (completamente esférica) a E7 (altamente achatada).

Galáxias Espirais: Essas galáxias possuem discos bem definidos, com braços espirais contendo regiões de formação estelar ativa, rodeadas por halos de estrelas antigas. As galáxias espirais são classificadas em tipos como Sa, Sb, e Sc, dependendo do grau de abertura e definição dos braços espirais. Há também uma subclasse de galáxias espirais barradas (SB), onde os braços emergem de uma barra central de estrelas.

Galáxias Irregulares: Essas galáxias não possuem uma estrutura definida ou simetria clara. Elas geralmente apresentam formações estelares caóticas e são frequentemente associadas a interações galácticas ou a processos de fusão. A morfologia irregular indica que a galáxia foi significativamente perturbada por interações gravitacionais com outras galáxias ou aglomerados de galáxias.

Morfologia e Evolução Galáctica

A morfologia de uma galáxia fornece informações fundamentais sobre sua história de formação e evolução. Galáxias elípticas, por exemplo, são frequentemente associadas a processos de fusão, onde galáxias menores colidem e se fundem, resultando em uma redistribuição das estrelas em uma forma mais esférica e homogênea. Já galáxias espirais tendem a manter seus discos bem definidos, com formação estelar sustentada ao longo do tempo, sugerindo que elas evoluem de forma mais lenta e estável.

O estudo das morfologias galácticas permite investigar como diferentes processos dinâmicos, como fusões, interações gravitacionais e acreção de matéria, influenciam a evolução estrutural das galáxias. Além disso, a distribuição de gás, poeira e regiões de formação estelar dentro de uma galáxia está diretamente relacionada à sua morfologia e aos processos físicos que determinam sua evolução.
Morfologia e Lentes Gravitacionais

A morfologia também desempenha um papel central em estudos de lentes gravitacionais, onde a forma das galáxias de fundo distorcidas pela presença de uma lente gravitacional (como aglomerados de galáxias) pode fornecer informações sobre a distribuição de massa na lente. A elipticidade e o alongamento das galáxias de fundo são analisados para reconstruir o potencial gravitacional da lente e, por extensão, a distribuição de matéria escura nessas regiões.
Importância dos Levantamentos Astronômicos

Com o advento de grandes levantamentos astronômicos, como o Sloan Digital Sky Survey (SDSS) e o Dark Energy Survey (DES), o estudo da morfologia galáctica atingiu uma escala sem precedentes. Esses levantamentos fornecem catálogos contendo milhões de galáxias com informações morfológicas detalhadas, permitindo uma análise estatística robusta de como a morfologia das galáxias varia com o tempo e o ambiente cosmológico.

Além disso, técnicas de aprendizado de máquina têm sido aplicadas para classificar galáxias em grande escala, automatizando o processo de classificação morfológica com base em dados de imagens. Modelos de redes neurais convolucionais, por exemplo, têm sido usados para identificar padrões morfológicos em galáxias, contribuindo para a classificação precisa de grandes volumes de dados astronômicos.
Morfologia em Cosmologia

A morfologia galáctica tem implicações diretas na cosmologia, especialmente no contexto da formação de estruturas em larga escala no universo. A análise de como a morfologia das galáxias varia com a densidade do ambiente, por exemplo, fornece pistas sobre a interação entre matéria escura, gás e galáxias no crescimento de aglomerados de galáxias. Galáxias elípticas tendem a ser mais comuns em aglomerados densos, enquanto galáxias espirais predominam em regiões de menor densidade.

A distribuição e evolução das morfologias galácticas em diferentes épocas cósmicas também oferecem insights sobre a história de formação estelar no universo e sobre a taxa de fusões galácticas. Estudos morfológicos de galáxias em diferentes redshifts (distâncias) permitem traçar uma cronologia detalhada da evolução galáctica, ajudando a validar ou refutar modelos teóricos de formação de galáxias.
Conclusão

A morfologia é uma ferramenta fundamental no estudo da astronomia, fornecendo informações sobre a estrutura, formação e evolução de galáxias e outros objetos celestes. A classificação morfológica e a análise das formas de galáxias em diferentes ambientes e épocas cósmicas têm sido essenciais para entender os processos dinâmicos que moldam o universo. Com o aumento do volume de dados astronômicos e o desenvolvimento de novas tecnologias de análise, a morfologia continuará desempenhando um papel central na astrofísica observacional e na cosmologia.



\section{Propriedades estruturais de uma galáxia}

\subsection{Raio petrosiano}

O raio Petrosiano (ou Petrosian Radius) é uma métrica utilizada para definir o tamanho de uma galáxia de forma que a medida seja menos dependente da distância do objeto e de sua inclinação. Ele é definido com base no perfil de luminosidade superficial de uma galáxia, isto é, a variação da intensidade da luz em função da distância ao centro da galáxia.

Mais formalmente, o raio Petrosiano $R_p$ é o raio no qual a razão entre a intensidade de superfície $I(r)$ em um dado raio $r$ e a intensidade média $.I.(r)$ dentro desse raio atinge um valor constante, geralmente escolhido como $\eta=0.2$:
% η(r)=I(r)⟨I⟩(r)=I(r)1r2∫0rI(r′) r′ dr′
% η(r)=⟨I⟩(r)I(r)​=r21​∫0r​I(r′)r′dr′I(r)​

Essa definição permite capturar a maioria da luz da galáxia sem ser excessivamente sensível a flutuações no brilho superficial nas regiões mais externas. O raio Petrosiano tem sido amplamente utilizado em levantamentos como o Sloan Digital Sky Survey (SDSS) para definir o tamanho aparente de galáxias de forma robusta.


\subsection{Raio Efetivo}
O raio $R_{50}$, ou raio efetivo (também chamado de Half-Light Radius), é outra métrica utilizada para descrever o tamanho de galáxias e outros objetos astronômicos. O $R_{50}$ é definido como o raio que contém 50% da luminosidade total do objeto, ou seja, metade da luz emitida pela galáxia está contida dentro desse raio.

% Matematicamente, se LtotLtot​ é a luminosidade total da galáxia, então o raio R50 R50R50​ é o raio no qual a luminosidade cumulativa L(r)L(r) atinge 0.5⋅Ltot0.5⋅Ltot​:
% L(R50)=0.5⋅Ltot
% L(R50​)=0.5⋅Ltot​

O raio $R_{50}$ é uma métrica importante porque fornece uma estimativa da concentração de luz em torno do centro da galáxia. Galáxias com morfologias elípticas, que tendem a ser mais concentradas, possuem valores de $R_{50}$ menores em comparação com galáxias espirais, que exibem discos mais extensos. Além disso, o raio $R_{50}$ é amplamente utilizado para descrever perfis de brilho superficial e estudar a evolução estrutural de galáxias ao longo do tempo.


\subsection{Elipticidade}
A elipticidade de um objeto astronômico é uma métrica geométrica que descreve o grau de achatamento ou alongamento de sua forma. É definida em termos da relação entre os eixos maior e menor de uma elipse que melhor descreve a forma do objeto projetado no céu. Se $a$ é o comprimento do eixo maior e $b$ é o comprimento do eixo menor, a elipticidade $\epsilon$ é dada por:
% ϵ=1−ba
% ϵ=1−ab​

% A elipticidade varia de 0 a 1, onde ϵ=0ϵ=0 corresponde a um objeto perfeitamente circular (com a=ba=b) e ϵ=1ϵ=1 corresponde a um objeto extremamente achatado, com b=0b=0.

Na astronomia, a elipticidade é uma característica importante para a classificação morfológica de galáxias. Galáxias elípticas tendem a ter uma elipticidade mais alta, enquanto galáxias espirais podem ter elipticidades menores, dependendo da inclinação de seu disco em relação à linha de visada. Além disso, a elipticidade pode fornecer informações sobre a dinâmica interna do objeto, como a presença de rotação ou efeitos gravitacionais de interações com outras galáxias.
Aplicações das Métricas no Estudo de Galáxias

Essas três métricas são essenciais para o estudo das propriedades estruturais de galáxias e têm diversas aplicações na pesquisa astronômica:

Classificação Morfológica: O raio Petrosiano e o raio R50 são amplamente utilizados para classificar galáxias de acordo com suas morfologias, distinguindo entre galáxias elípticas, espirais e irregulares. O estudo desses parâmetros permite entender como as galáxias evoluem ao longo do tempo, desde estruturas mais compactas até discos expandidos.

Estudos de Concentração: O raio R50 é frequentemente usado em combinação com o raio que contém 90\% da luz (R90) para calcular a concentração de uma galáxia, uma métrica que descreve a distribuição da luz. Galáxias com uma concentração maior tendem a ser mais compactas e massivas, enquanto galáxias com uma concentração menor possuem discos mais extensos.

Dinâmica Interna e Formação de Galáxias: A elipticidade fornece pistas sobre a dinâmica e a formação de galáxias. Por exemplo, uma alta elipticidade pode indicar que a galáxia passou por fusões recentes ou está interagindo gravitacionalmente com outras galáxias próximas. Em contraste, galáxias mais circulares geralmente apresentam uma evolução mais tranquila, sem interações recentes.

Levantamentos Astronômicos: Em levantamentos como o Sloan Digital Sky Survey (SDSS), tanto o raio Petrosiano quanto o raio R50 e a elipticidade são rotineiramente medidos para milhões de galáxias, permitindo uma análise estatística robusta das propriedades estruturais de grandes populações de galáxias em diferentes épocas cosmológicas.



\section{Recuperação de Imagens Baseada em Conteúdo}
O Content-Based Image Retrieval (CBIR), ou Recuperação de Imagens Baseada em Conteúdo, é uma técnica amplamente utilizada para realizar a busca e recuperação de imagens em grandes bases de dados de maneira automática, utilizando características visuais extraídas diretamente das imagens. Ao contrário dos métodos tradicionais, que utilizam metadados textuais para indexação, o CBIR foca em aspectos intrínsecos das imagens, como textura, forma, cor e padrões, para encontrar similaridades entre imagens diferentes.

A técnica de CBIR envolve vários estágios, como a extração de características visuais, a construção de representações vetoriais das imagens, a definição de uma métrica de similaridade para comparar essas representações e, finalmente, a recuperação de imagens similares com base na consulta fornecida. Entre os algoritmos mais utilizados para a extração de características visuais, destacam-se redes neurais convolucionais (CNNs), que aprenderam representações visuais de alto nível em diversas tarefas, incluindo classificação e detecção de objetos.

A importância do CBIR na astronomia tem crescido substancialmente nas últimas décadas devido à explosão de dados gerados por grandes levantamentos astronômicos, como o Sloan Digital Sky Survey (SDSS) e o próximo Legacy Survey of Space and Time (LSST). Esses levantamentos produzem bilhões de imagens de objetos celestes, como estrelas, galáxias e nebulosas, criando a necessidade de métodos automáticos e eficientes para encontrar e categorizar esses objetos com base em suas características visuais.

Um exemplo de aplicação do CBIR na astronomia é a busca por galáxias com formas similares em imagens capturadas por telescópios. A classificação visual de galáxias é um problema fundamental, pois diferentes morfologias podem estar associadas a diferentes processos físicos e evolutivos. O uso de técnicas de CBIR possibilita a busca automática de galáxias que compartilham características morfológicas semelhantes, como espirais ou elípticas, o que pode auxiliar os astrônomos na descoberta de padrões e na categorização de objetos celestes de maneira mais eficiente.

Além disso, o CBIR facilita a busca por eventos astronômicos raros ou peculiares. Muitas vezes, eventos transitórios, como supernovas ou lentes gravitacionais, possuem assinaturas visuais específicas, e o CBIR pode auxiliar na detecção e no agrupamento desses eventos, tornando o processo de descoberta mais rápido e preciso.

A técnica de CBIR também desempenha um papel fundamental em sistemas inteligentes baseados em aprendizado profundo, uma vez que as redes neurais são capazes de aprender representações visuais discriminativas diretamente a partir dos dados. Ao treinar modelos com imagens astronômicas anotadas, é possível desenvolver sistemas capazes de identificar automaticamente objetos com alta precisão, utilizando métodos de similaridade visual para recuperar objetos de interesse.

Em resumo, a técnica de CBIR é crucial para a astronomia moderna, permitindo a exploração eficiente e escalável de enormes quantidades de dados visuais, otimizando a busca por objetos e eventos astronômicos e contribuindo para avanços em pesquisa e descobertas científicas.


\section{Ciência Cidadã}

A Ciência Cidadã (citizen science) refere-se à participação ativa de cidadãos comuns, que não necessariamente possuem formação científica, em projetos de pesquisa conduzidos por cientistas profissionais. Esses cidadãos contribuem voluntariamente em várias etapas do processo científico, como coleta, análise de dados, ou mesmo formulação de hipóteses. A Ciência Cidadã tem se expandido consideravelmente nas últimas décadas, impulsionada principalmente pelos avanços tecnológicos e pela crescente disponibilidade de plataformas digitais que facilitam a interação entre cientistas e o público. Essa abordagem é caracterizada por seu caráter colaborativo, permitindo que o conhecimento científico seja construído de forma coletiva, aproveitando o potencial de um grande número de participantes distribuídos geograficamente.

A contribuição da Ciência Cidadã para o progresso científico é significativa, particularmente em áreas que demandam grandes quantidades de dados. Projetos que envolvem a observação da natureza, como a ecologia, a biologia e a astronomia, se beneficiam amplamente dessa participação. Um exemplo relevante é o projeto Galaxy Zoo, onde cidadãos ajudam a classificar imagens de galáxias obtidas por telescópios, um processo que seria inviável para ser feito exclusivamente por pesquisadores, dada a quantidade massiva de imagens coletadas. De forma semelhante, o projeto eBird tem revolucionado o estudo das aves ao permitir que observadores do mundo todo registrem avistamentos de espécies, criando um banco de dados global que seria impossível de ser gerado apenas por cientistas profissionais.

Além de aumentar significativamente a capacidade de coleta e processamento de dados, a Ciência Cidadã desempenha um papel importante no engajamento público com a ciência. Ao participar ativamente de projetos científicos, os cidadãos desenvolvem uma melhor compreensão sobre os métodos e processos envolvidos na pesquisa científica. Isso não apenas melhora a alfabetização científica da população, mas também promove uma conexão mais forte entre a sociedade e a ciência. A Ciência Cidadã pode, portanto, gerar uma maior confiança pública na ciência, além de inspirar novos interesses e até carreiras em áreas científicas.

Outro aspecto relevante da Ciência Cidadã é a possibilidade de gerar descobertas científicas inesperadas. Em alguns casos, voluntários têm identificado fenômenos ou padrões que passaram despercebidos por cientistas. Por exemplo, no contexto da astronomia, um voluntário no Galaxy Zoo identificou uma nova classe de galáxias, denominada "Peas", levando a novos estudos sobre a formação estelar. Esses exemplos mostram como a colaboração entre cientistas e o público pode levar a avanços significativos no conhecimento científico.

No entanto, a Ciência Cidadã também enfrenta desafios. A qualidade dos dados coletados por voluntários pode variar, exigindo que os cientistas implementem métodos robustos de validação e calibração. Além disso, é necessário que os projetos ofereçam treinamento adequado aos participantes, para garantir que suas contribuições sejam precisas e cientificamente válidas. Outro desafio está relacionado à inclusão, já que o acesso a tecnologias e à educação científica pode limitar a participação de certos grupos populacionais.

Dessa forma, a Ciência Cidadã é uma ferramenta poderosa para ampliar o alcance e o impacto da pesquisa científica. Ela permite que cientistas coletem e analisem dados em escalas que seriam impraticáveis de outra forma, ao mesmo tempo em que engaja e educa o público sobre o processo científico. Com uma participação ativa e colaborativa, a Ciência Cidadã contribui não apenas para o avanço do conhecimento em diversas áreas, mas também para a criação de uma sociedade mais informada e envolvida com a ciência.
