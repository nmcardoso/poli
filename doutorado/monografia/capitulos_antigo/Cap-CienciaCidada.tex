\chapter{Ciência Cidadã}
\label{cap:cs}

\begin{overview}
  \lipsum[1]
\end{overview}


\section{Introdução}
A ciência cidadã é um campo interdisciplinar emergente que envolve a participação ativa do público geral em tarefas de pesquisas científicas com finalidade de produzir novos conhecimentos para a ciência e para a sociedade \cite{scs-1}, especialmente na coleta, categorização, transcrição e análise de dados científicos \cite{silvertown2009,bonney2014}. Este modelo de pesquisa tem ganhado relevância em diversos campos da ciência, principalmente devido ao aumento do acesso a tecnologias digitais e à internet, que facilitam a comunicação e a organização entre cidadãos e cientistas \cite{scs-4}.
%No decorrer dessa seção, será feito um breve levantamento do uso da ciência cidadã em diferentes áreas do conhecimento (Seção \ref{sec:cs-areas}), bem como na astronomia (Seção \ref{sec:cs-astro}), e uma perspectiva sobre a qualidade dos dados gerados (Seção \ref{sec:cs-data-quality}).

\section{GalaxyZoo}
Na astronomia, o GalaxyZoo\footnote{\url{https://galaxyzoo.org}} \cite{gz} é um projeto de ciência cidadã lançado em 2007, cujo objetivo é classificar morfologicamente galáxias utilizando imagens obtidas por grandes levantamentos astronômicos, como o Sloan Digital Sky Survey (SDSS). Os participantes, voluntários de diferentes formações e níveis de conhecimento científico, analisam imagens de galáxias e fornecem informações sobre suas características, como forma espiral, elíptica ou irregular, e a presença de estruturas específicas, como barras centrais. Essa colaboração massiva permitiu a classificação de milhões de galáxias em um curto período, superando em eficiência o que seria possível com equipes científicas tradicionais.

O impacto científico do GalaxyZoo é expressivo. Além de fornecer uma base de dados robusta e de alta qualidade para a pesquisa astronômica, o projeto gerou avanços no entendimento da formação e evolução de galáxias, incluindo a relação entre morfologia e ambiente. Os resultados têm sido utilizados para treinar modelos de aprendizado de máquina, permitindo a automação de tarefas de classificação em levantamentos futuros. O sucesso do GalaxyZoo também inspirou o desenvolvimento de outras plataformas de ciência cidadã, consolidando seu papel como ferramenta metodológica na pesquisa científica.

Para os voluntários, o GalaxyZoo oferece uma oportunidade de engajamento direto com a ciência, promovendo aprendizado e um senso de contribuição para descobertas científicas relevantes. Muitos participantes relatam um aumento na compreensão de conceitos astronômicos e motivação para explorar outras áreas da ciência. Além disso, a plataforma fomenta uma comunidade global de entusiastas que colaboram ativamente, monstando como iniciativas bem estruturadas podem democratizar o acesso e a participação no progresso científico.



% \subsection{Áreas de Atuação da Ciência Cidadã}
% \label{sec:cs-areas}
% \lipsum[4]


% \subsection{Ciência Cidadã na Astronomia}
% \label{sec:cs-astro}
% \lipsum[4]


% \subsection{Qualidade dos Dados}
% \label{sec:cs-data-quality}
% \lipsum[4]

\chaptersep
