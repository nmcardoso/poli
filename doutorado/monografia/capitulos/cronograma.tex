\chapter{Cronograma}
\label{cap:cronograma}

\begin{overview}
  Este capítulo mostra o cronograma de atividades do projeto dividido em duas partes: o \textbf{cronograma principal} (Seção \ref{sec:cronograma-principal}), que mostra as atividades planejadas para o atual curso de mestrado, e o \textbf{cronograma extendido} (Seção \ref{sec:cronograma-extendido}), que mostra uma extensão do cronograma principal com a inclusão das ativadades planejadas para um curso de doutorado.
\end{overview}


\section{Cronograma principal}
\label{sec:cronograma-principal}
\begin{enumerate}
  \item \textbf{Preparação do conjunto de treinamento} (2 meses, concluído) -- \textit{aquisição e preparação do conjunto de referência GalaxyZoo.}
  \item \textbf{Aquisição de imagens RGB} (2 meses, concluído) -- \textit{coleta de imagens RGB do Legacy Survey dos conjuntos de referência e inferência.}
  \item \textbf{Modelo de aprendizado profundo com imagens RGB} (2 meses, concluído) -- \textit{implementação, treinamento e avaliação do modelo.}
  \item \textbf{Modelo proposto} (2 meses, concluído) -- \textit{implementação, treinamento e avaliação do modelo proposto, utilizando Spectral Positional Encoding.}
  \item \textbf{Matérias da pós-graduação} (6 meses, paralelo, concluído) -- \textit{cursar as metérias da pós-graduação, mais informações na Seção \ref{sec:disciplinas}}.
\end{enumerate}


\section{Cronograma extendido}
\label{sec:cronograma-extendido}
\begin{enumerate}
  \item \textbf{Avaliação e seleção dos principais levantamentos astronômicos} (1 mes, concluído) -- \textit{analise de fatores como cobertura espacial e espectral, qualidade dos dados (exemplo: profundidade, resolução angular e razão sinal-ruido da imagem) e padronização do acesso aos dos dados para criar um conjunto de fontes}.
  \item \textbf{Aquisição dos dados} (4 meses, concluído) -- \textit{implementação de clientes para coleta de dados (imagens e tabelas) de diferentes observatórios selecionados na etapa 1}.
  \item \textbf{Modelo de aprendizado profundo com imagens multiespectrais} (2 meses, parcialmente concluído) -- \textit{implementação, treinamento e avaliação de um modelo de aprendizado profundo para processamento de imagens multiespectrais utilizando técnicas conhecidas da literatura.}
  \item \textbf{Modelo de aprendizado profundo com imagens multiespectrais} (4 meses) -- \textit{implementação, treinamento e avaliação de um modelo de aprendizado profundo para processamento de imagens multiespectrais utilizando o método proposto por este trabalho: Spectral Positional Encoding.}
  \item \textbf{} -- \textit{comparação entre os modelos base-}.
\end{enumerate}


\chaptersep
